% ==============================================================================
% LAB 167
% MÄTNING PÅ ELEKTRISKA KRETSAR
% --------------------------
% Last updated <2015-02-22> 
%
% Author:
% Jonas Sjöberg     <tel12jsg@student.hig.se>
% Oscar Wallberg    <tco13owg@student.hig.se>
% 
% License:
% Creative Commons Attribution-NonCommercial-ShareAlike 4.0 International
% See LICENSE.md for full licensing information.
% ==============================================================================

% ==============================================================================
% INCLUDES AND CONFIGURATION
% ==============================================================================
\documentclass[11pt,a4paper]{article}
\usepackage[utf8]{inputenc}
\usepackage[swedish]{babel} % För svensk innehållsförteckning
\usepackage{siunitx} % (För dokumentation, kör i terminalen; texdoc siunitx)
\usepackage{amssymb}
\usepackage{amsmath}
\usepackage{amsfonts}
\usepackage{graphicx}
\usepackage{booktabs}
\usepackage{longtable} % Tables span across pages
\usepackage{microtype}
\usepackage{gensymb}
%\usepackage{tabto}
\usepackage{units}

\setlength\parindent{0pt} % Removes all indentation from paragraphs

% ==============================================================================
% DOCUMENT METADATA 
% ==============================================================================
\title{EE466 \\ Lab 167 \\ Mätning på elektriska kretsar}

\author{\\
  Jonas Sjöberg\\
  Högskolan i Gävle,\\
  Elektronikingenjörsprogrammet,\\
  \texttt{tel12jsg@student.hig.se}\\
  \\
  Oscar Wallberg\\
  Högskolan i Gävle,\\
  Dataingenjörsprogrammet,\\
  \texttt{tco13owg@student.hig.se}\\}

\date{}
% ==============================================================================
\begin{document}
% ==============================================================================
\maketitle

\begin{center}
    \begin{tabular}{l r}
        Labb utförd: & ? Februari 2015 \\
        Instruktör: & Efrain Zenteno
    \end{tabular}
\end{center}

% ==============================================================================
% ABSTRACT
% ==============================================================================
\begin{abstract}
    Syftet med laborationen är att praktiskt pröva några av de grundläggande
    sambanden och satserna i likströmsläran, samt att förstå enkla
    växelströmskretsar. Dessutom bör studenten efter genomförd laboration
    översiktligt förstå universalinstrumentets och oscilloskopets principiella
    funktionssätt, samt kunna tillämpa hanteringen av dessa instrument i
    mätning på elektriska kretsar.
\end{abstract}

\newpage

{
    %\hypersetup{linkcolor=black}
    \setcounter{tocdepth}{3}
    \tableofcontents
}

\newpage

% ==============================================================================
% SECTION: INTRODUKTION 
% ==============================================================================
\section{Introduktion}\label{setup}
% ==============================================================================
% TODO: Allmän introduktion.


% ==============================================================================
% SECTION: 1 MÄTNING PÅ SERIESKRETS
% ==============================================================================
\section{Mätning på seriekrets}\label{}
% ==============================================================================
% TODO: Kopplingsschema.
Seriekretsen enligt figur \ref{fig:1-mm-schem} kopplades upp. 5 \si{\volt} valdes för spänningskällan.
\begin{figure}[htbp]
    \centering
        \includegraphics[scale=0.7]{misc/krets1.png}
    \caption{Seriekrets}
    \label{fig:1-mm-schem}
\end{figure}
\subsection{Mätresultat}\label{}
% ------------------------------------------------------------------------------
% TODO: + Spänningarna mellan AB, BC och AC.
Resistensen mellan A och B, $R_1$, mättes upp till $100.561 \ohm$ och mellan B och C, $R_2$, mättes $217.78 \ohm$ upp. Följande spänningar mättes därefter upp:
\begin{math}
U_{AB} = 1.58 \si{\volt}\\
U_{BC} = 3.41 \si{\volt}\\
U_{AC} = 4.999 \si{\volt}
\end{math}
\subsection{Kommentar}\label{}
% ------------------------------------------------------------------------------
% TODO: Kommentera utgående från Kirchhoffs 2:a lag.
%       Kommentera utgående från spänningsdelningslagen.
Spänningsdelningslagen ger:\\[+2mm]
\begin{math}
U_{AB} = U\times\frac{R_{1}}{R_{1}+R_{2}}\\[+2mm]
U_{AB} = 4.999\times\frac{100.561}{100.561+217.78}\\[+2mm]
U_{AB} = 1.579\\
\\
U_{BC} = U\times\frac{R_{2}}{R_{1}+R_{2}}\\[+2mm]
U_{BC} = 4.999\times\frac{217.78}{100.561+217.78}\\[+2mm]
U_{BC} = 3.42\\
\\
U_{AC} = U\times\frac{R_{1}+R_{2}}{R_{1}+R_{2}}\\[+2mm]
U_{AC} = U = 4.999\\
\end{math}

Kirchhoff's 2:a lag:
\begin{quote}
Summan av samtliga emk:s som ingår i en sluten krets är lika med summan av potentialfallen, eller\\
\begin{math}
u_{1} + u_{2} + \ldots + u_{n} = 0\\
\text{där }u_{k} \text{ betecknar en potentialändring.}
\end{math}
\end{quote}

Enligt Kirchhoff's lag:\\
$U - U_{AB} - U_{BC} = 0$\\
$4.99 - 1.579 - 3.42 = 0$, vilket stämmer.
% ==============================================================================
% SECTION: 2 iNVERKAN AV EN PARALLELLGREN PÅ EN KRETS
% ==============================================================================
\section{Inverkan av en parallellgren på en krets}\label{}
% ==============================================================================
% TODO: Kopplingsschema.

\subsection{Mätresultat}\label{}
% ------------------------------------------------------------------------------
% TODO: Mät strömmen i punkten B samt strömmen direkt från spänningskällan.


% ==============================================================================
% SECTION: 3 MÄTNING PÅ PARALLELLKRETS
% ==============================================================================
\section{Mätning på parallellkrets}\label{}
% ==============================================================================
% TODO: Kopplingsschema.

\subsection{Mätresultat}\label{}
% ------------------------------------------------------------------------------
% TODO: Mät de markerade strömmarna

\subsection{Kommentar}\label{}
% ------------------------------------------------------------------------------
% TODO: Kommentera utgående från Kirchhoffs 1:a lag.


% ==============================================================================
% SECTION: 4 MÄTNING AV RESISTANS
% ==============================================================================
\section{Mätning av resistans}\label{}
% ==============================================================================
% TODO: Kopplingsschema.

\subsection{Mätresultat}\label{}
% ------------------------------------------------------------------------------
% TODO: Mät resistansen mellan A och B i nedanstående kretsar.

\subsection{Teoretisk beräkning}\label{}
% ------------------------------------------------------------------------------
% TODO: För varje mätning skall du verifiera resultatet med en teoretisk beräkning.

%\subsection{Kommentar}\label{}
% ------------------------------------------------------------------------------
% TODO: Kommentar på skillnader mellan mätresultat och beräkning?


% ==============================================================================
% SECTION: 5 MÄTNING AV EMK OCH INRE RESISTANS I EN TVÅPOL
% ==============================================================================
\section{Mätning av emk och inre resistans i en tvåpol}\label{}
% ==============================================================================
% TODO: Mät tvåpolens tomgångsspänning, Uab0, d.v.s. ställ Ry på sitt maximala värde.
%       Minska Ry tills spänningen över Ry har minskat till 0.5.Uab0 (d.v.s hälften av
%       tomgångsspänningen).
%
%       * Värdet på Ry då tvåpolens spänning har sjunkit till hälften är lika stort som
%         tvåpolens inre resistans, Ri. Utred varför!
%       * Härled med Thévenins teorem de teoretiska värden på Uab0 och Ri.


% ==============================================================================
% SECTION: 6 KARAKTERISTIK HOS EN LYSDIOD
% ==============================================================================
\section{Karakteristik hos en lysdiod}\label{}
% ==============================================================================
% TODO: Rita grafen I=f(U)! Kommentera!


% ==============================================================================
% SECTION: 7 MÄTNING AV VÄXELSPÄNNING MED UNIVERSALINSTRUMENT OCH OSCILLOSKOP
% ==============================================================================
\section{Mätning av växelspänning med universalinstrument och oscilloskop}\label{}
% ==============================================================================
% TODO


% ==============================================================================
% SECTION: 8 STUDIUM AV FREKVENSGÅNG I EN REAKTIV KRETS
% ==============================================================================
\section{Studium av frekvensgång i en reaktiv krets}\label{}
% ==============================================================================
% TODO: Kopplingsschema.

\subsection{Mätresultat}\label{}
% ------------------------------------------------------------------------------
% TODO: + Gör upp en tabell som för varje frekvens anger
%         - tongeneratorns signalamplitud,
%         - amplituden hos spänningen över den studerade kondensatorn
%         - kvoten mellan den senare amplituden och den tidigare
%         - samt om fasförskjutning förekommer.

\subsection{Teoretisk beräkning}\label{}
% ------------------------------------------------------------------------------
% TODO: Kontrollera dina resultat genom att utnyttja följande formel:

\subsection{Kommentar}\label{}
% ------------------------------------------------------------------------------
% TODO:


% ==============================================================================
% SECTION: 9 MÄTNING AV FASFÖRSKJUTNING I EN REAKTIV KRETS
% ==============================================================================
\section{Mätning av fasförskjutning i en reaktiv krets}\label{}
% ==============================================================================
% TODO: Kopplingsschema.
%       Samma koppling som förra uppgiften, kanske överflödigt att upprepa?

\subsection{Mätresultat}\label{}
% ------------------------------------------------------------------------------
% TODO:

\subsection{Teoretisk beräkning}\label{}
% ------------------------------------------------------------------------------
% TODO: Kontrollera dina resultat genom att utnyttja följande formel:

\subsection{Kommentar}\label{}
% ------------------------------------------------------------------------------
% TODO: Kommentera resultatet


% ==============================================================================
% SECTION: 10 MÄTNING AV RESONANSFREKVENS
% ==============================================================================
\section{Mätning av resonansfrekvens}\label{}
% ==============================================================================
% TODO: Kopplingsschema.

\subsection{Mätresultat}\label{}
% ------------------------------------------------------------------------------
% TODO: Notera resonansfrekvensen. 

\subsection{Kommentar}\label{}
% ------------------------------------------------------------------------------
% TODO: + Kommentera följande:                                                            
%          - Förekommer fasförskjutning mellan uTG och uR vid denna frekvens?
%          - Ändrar fasen sig om du varierar frekvensen kring den
%            uppmätta resonansfrekvensen?  I så fall hur?
%          - Om laboration 61 har gjorts, jämför resultatet med de uppmätta
%            värderna.  Stämmer de överens om inte, varför?


% ==============================================================================
% SECTION: RESULTAT
% ==============================================================================
\section{Resultat}\label{setup}
% ==============================================================================
% TODO: Övergripande resultat/sammanfattning/kommentar på HELA labben.

\newpage

% ==============================================================================
% SECTION: REFERENSER
% ==============================================================================
\section{Referenser}\label{refs}
% ==============================================================================
%TODO: Referenser.

%\subsection{www}\label{interwebs}
% ------------------------------------------------------------------------------

%\subsection{Trycksaker}\label{literature} %???
% ------------------------------------------------------------------------------

%\subsection{Källkod}\label{sourcefiles}
% ------------------------------------------------------------------------------

% ==============================================================================
\end{document}
% ==============================================================================
